%%%%%%%%%%%%%%%%%%%%%%%%%%%%%%%%%%%%%%%%%%%%%%%%%%%%%%
% A Beamer template for HKUST                        %
% Based on THU beamer theme                          %
% Author: Yuxuan HU                                  %
% Date: Aug 2024                                     %
% LPPL Licensed.                                     %
%%%%%%%%%%%%%%%%%%%%%%%%%%%%%%%%%%%%%%%%%%%%%%%%%%%%%%

\documentclass[serif, aspectratio=169]{beamer}
%\documentclass[serif]{beamer}  % for 4:3 ratio
\usepackage[utf8]{inputenc}
\usepackage[T1]{fontenc} 
\usepackage{fourier} % see "http://faq.ktug.org/wiki/uploads/MathFonts.pdf" for other options
\usepackage{hyperref}
\usepackage{latexsym,amsmath,xcolor,multicol,booktabs,calligra}
\usepackage{graphicx,pstricks,listings,stackengine}
\usepackage{lipsum}

\author{Zirui Zhang}
\title{Java Syntax \& Data Types}
\subtitle{Programming with Java: Session 2}
\institute{
    Tornado Engineering Club \\
    Jinan Foreign Language School
}
\date{\small \today}
\usepackage{style}
\definecolor{codegreen}{rgb}{0,0.6,0}
\definecolor{codegray}{rgb}{0.5,0.5,0.5}
\definecolor{codepurple}{rgb}{0.58,0,0.82}
\definecolor{backcolour}{rgb}{0.95,0.95,0.92}

\lstdefinestyle{mystyle}{
    backgroundcolor=\color{backcolour},   
    commentstyle=\color{codegreen},
    keywordstyle=\color{magenta},
    numberstyle=\tiny\color{codegray},
    stringstyle=\color{codepurple},
    basicstyle=\ttfamily\footnotesize,
    breakatwhitespace=false,         
    breaklines=true,                 
    captionpos=b,                    
    keepspaces=true,                 
    numbers=left,                    
    numbersep=5pt,                  
    showspaces=false,                
    showstringspaces=false,
    showtabs=false,                  
    tabsize=2
}

\lstset{style=mystyle}

% defs
\def\cmd#1{\texttt{\color{red}\footnotesize $\backslash$#1}}
\def\env#1{\texttt{\color{blue}\footnotesize #1}}
% set colors
\definecolor{hkustyellow}{RGB}{167, 131, 55}
\definecolor{hkustblue}{RGB}{85, 172, 238}
\definecolor{hkustred}{RGB}{209, 51, 59}
\definecolor{halfgray}{rgb}{0.5, 0.5, 0.5}


%- --- --- --- --- --- --- --- --- --- --- --- --- --- --- --- 
\begin{document}

\begin{frame}
    \titlepage
    \vspace*{-0.4cm} % Adjusted to avoid overfull \vbox
    \begin{figure}[htpb]
        \begin{center}
            \includegraphics[keepaspectratio, width=2.5cm]{pic/tornado.png}
        \end{center}
    \end{figure}
    \vspace{-1.5em}
    \courtesy{https://liaoxuefeng.com/books/java/}
\end{frame}

\begin{frame}    
\tableofcontents[sectionstyle=show,
subsectionstyle=show/shaded/hide,
subsubsectionstyle=show/shaded/hide]
\end{frame}

\section{Basic Structure}
\begin{frame}[fragile]{Complete Java Program Example}
\begin{lstlisting}[language=Java]
/**
 * Comments for documentation generation
 */
public class Hello {
    public static void main(String[] args) {
        // Output text to the screen:
        System.out.println("Hello, world!");
        /* Multi-line comment start
        Comment content
        Comment end */
    }
} // End of class definition
\end{lstlisting}
\end{frame}

\begin{frame}[fragile]{Class Structure in Java}
\begin{itemize}
    \item Java is object-oriented → basic unit is a \texttt{class}
    \item \texttt{class} is a keyword
    \item Class name follows specific naming conventions
\end{itemize}

\begin{lstlisting}[language=Java]
public class Hello { // Class name is Hello
    // Class body...
} // End of class definition
\end{lstlisting}
\end{frame}

\begin{frame}[fragile]{Class Naming Conventions}
\begin{block}{Good Class Names:}
\begin{itemize}
    \item \texttt{Hello}
    \item \texttt{NoteBook} 
    \item \texttt{VRPlayer}
\end{itemize}
\end{block}

\begin{block}{Bad Class Names:}
\begin{itemize}
    \item \texttt{hello} (should start with uppercase)
    \item \texttt{Good123} (meaningless numbers)
    \item \texttt{Note\_Book} (avoid underscores)
    \item \texttt{\_World} (should not start with underscore)
\end{itemize}
\end{block}
\end{frame}

\begin{frame}[fragile]{Access Modifiers}
\begin{itemize}
    \item \texttt{public} indicates the class is accessible everywhere
    \item If omitted: program compiles but cannot be executed from command line
    \item Classes can contain multiple methods
\end{itemize}

\begin{lstlisting}[language=Java]
public class Hello {
    public static void main(String[] args) {
        // Method code...
    }
}
\end{lstlisting}
\end{frame}

\begin{frame}[fragile]{Method Structure}
\begin{itemize}
    \item Methods define execution statements
    \item Code executes sequentially
    \item \texttt{main} method is the program entry point
    \item \texttt{void} indicates no return value
    \item \texttt{static} indicates a static method
\end{itemize}

\begin{lstlisting}[language=Java]
public static void main(String[] args) {
    System.out.println("Hello, world!");
}
\end{lstlisting}
\end{frame}

\begin{frame}[fragile]{Method Naming Conventions}
\begin{block}{Good Method Names:}
\begin{itemize}
    \item \texttt{main}
    \item \texttt{goodMorning}
    \item \texttt{playVR}
\end{itemize}
\end{block}

\begin{block}{Bad Method Names:}
\begin{itemize}
    \item \texttt{Main} (should start with lowercase)
    \item \texttt{good123} (meaningless numbers)
    \item \texttt{good\_morning} (avoid underscores)
    \item \texttt{\_playVR} (should not start with underscore)
\end{itemize}
\end{block}
\end{frame}

\begin{frame}[fragile]{Statements and Syntax}
\begin{itemize}
    \item Statements are executable code
    \item Each statement must end with a semicolon
    \item Statements execute in sequence
\end{itemize}

\begin{lstlisting}[language=Java]
public class Hello {
    public static void main(String[] args) {
        System.out.println("Hello, world!"); // Statement
        int x = 5; // Another statement
    }
}
\end{lstlisting}
\end{frame}

\begin{frame}[fragile]{Comments in Java}
\begin{block}{Single-line Comments:}
\begin{lstlisting}[language=Java]
// This is a single-line comment
System.out.println("Hello"); // Comment after code
\end{lstlisting}
\end{block}

\begin{block}{Multi-line Comments:}
\begin{lstlisting}[language=Java]
/*
This is a multi-line comment
that spans multiple lines
*/
\end{lstlisting}
\end{block}
\end{frame}

\begin{frame}[fragile]{Documentation Comments}
\begin{itemize}
    \item Special multi-line comments for documentation
    \item Start with \texttt{/**} and end with \texttt{*/}
    \item Used to generate automatic documentation
    \item Place before class/method definitions
\end{itemize}

\begin{lstlisting}[language=Java]
/**
 * Class description for documentation
 * 
 * @author Developer Name
 * @version 1.0
 */
public class Hello {
    // Class implementation...
}
\end{lstlisting}
\end{frame}

\begin{frame}[fragile]{Code Formatting}
\begin{itemize}
    \item Java is flexible with whitespace
    \item Extra spaces/line breaks don't affect compilation
    \item Follow community coding conventions
    \item Use IDE formatting tools (Eclipse: \texttt{Ctrl+Shift+F})
    \item Consistent formatting improves readability
\end{itemize}

\begin{block}{IDE Settings:}
\texttt{Java → Code Style} in Eclipse preferences
\end{block}
\end{frame}

\begin{frame}[fragile]{Summary}
\begin{itemize}
    \item Java programs are organized in classes
    \item \texttt{main} method is the entry point
    \item Follow naming conventions for classes and methods
    \item Use proper comments for documentation
    \item Maintain consistent code formatting
    \item Statements must end with semicolons
\end{itemize}
\end{frame}

\section{Variables \& Data Types}
\begin{frame}[fragile]
\frametitle{What is a Variable?}
\begin{itemize}
    \item A concept from algebra (e.g., `x`, `y` in equations).
    \item In Java, variables are divided into two types:
    \begin{itemize}
        \item Primitive type variables
        \item Reference type variables
    \end{itemize}
\end{itemize}
\begin{lstlisting}[language=Java]
// Example equation in algebra:
y = x^2 + 1
\end{lstlisting}
\end{frame}

\begin{frame}[fragile]
\frametitle{Defining Variables}
\begin{itemize}
    \item Variables must be defined before use.
    \item Can assign an initial value during definition.
    \item If no initial value, assigned a default value (usually `0`).
\end{itemize}
\begin{lstlisting}[language=Java]
int x = 1; // Define an int variable x with initial value 1

// Example: Define and print a variable
public class Main {
    public static void main(String[] args) {
        int x = 100;
        System.out.println(x); // Prints 100
    }
}
\end{lstlisting}
\end{frame}

\begin{frame}[fragile]
\frametitle{Reassigning Variables}
\begin{itemize}
    \item Variables can be reassigned new values.
    \item Do not specify the type again when reassigning.
\end{itemize}
\begin{lstlisting}[language=Java]
public class Main {
    public static void main(String[] args) {
        int x = 100; // Define with initial value
        System.out.println(x); // Prints 100
        x = 200;     // Reassign (no 'int' here)
        System.out.println(x); // Prints 200
    }
}
\end{lstlisting}
\end{frame}

\begin{frame}[fragile]
\frametitle{Assigning Between Variables}
\begin{lstlisting}[language=Java]
public class Main {
    public static void main(String[] args) {
        int n = 100;
        System.out.println("n = " + n); // n = 100

        n = 200;
        System.out.println("n = " + n); // n = 200

        int x = n; // Assign n's value (200) to x
        System.out.println("x = " + x); // x = 200

        x = x + 100; // x becomes 300
        System.out.println("x = " + x); // x = 300
        System.out.println("n = " + n); // n = 200
   }
}
\end{lstlisting}
\end{frame}

\begin{frame}[fragile]
\frametitle{Memory Representation}
    % Variables correspond to memory locations and assignment writes a value to that location.
    \vspace{-1.8em}
    \begin{columns}
    \begin{column}{0.5\textwidth}
\begin{verbatim}
Step 1: int n = 100;
      n
      |
      v
+---+---+---+---+---+---+---+
|   |100|   |   |   |   |   |
+---+---+---+---+---+---+---+

Step 2: n = 200;
      n
      |
      v
+---+---+---+---+---+---+---+
|   |200|   |   |   |   |   |
+---+---+---+---+---+---+---+
\end{verbatim}

    \end{column}
    \begin{column}{0.5\textwidth}
\begin{verbatim}
Step 3: int x = n;
      n           x
      |           |
      v           v
+---+---+---+---+---+---+---+
|   |200|   |   |200|   |   |
+---+---+---+---+---+---+---+

Step 4: x = x + 100;
      n           x
      |           |
      v           v
+---+---+---+---+---+---+---+
|   |200|   |   |300|   |   |
+---+---+---+---+---+---+---+
\end{verbatim}
    \end{column}
    \end{columns}
\end{frame}

\begin{frame}[fragile]
\frametitle{Primitive Data Types}
\begin{itemize}
    \item Types the CPU can directly operate on.
    \item Java defines:
    \begin{itemize}
        \item Integer types: `byte`, `short`, `int`, `long`
        \item Floating-point types: `float`, `double`
        \item Character type: `char`
        \item Boolean type: `boolean`
    \end{itemize}
\end{itemize}
\begin{center}
% \includegraphics[width=0.8\textwidth]{data-types-memory.png} % Note: You would need to create this image
\end{center}
\end{frame}

\begin{frame}[fragile]
\frametitle{Integer Types}
\begin{itemize}
    \item Ranges:
    \begin{itemize}
        \item `byte`: -128 ~ 127
        \item `short`: -32768 ~ 32767
        \item `int`: -2147483648 ~ 2147483647
        \item `long`: -9223372036854775808 ~ 9223372036854775807
    \end{itemize}
\end{itemize}
\begin{lstlisting}[language=Java]
int i = 2147483647;
int i2 = -2147483648;
int i3 = 2_000_000_000; // Underscores for readability
int i4 = 0xff0000;      // Hex: 16711680
int i5 = 0b1000000000;  // Binary: 512

long n1 = 9000000000000000000L; // Requires L suffix
long n2 = 900;        // OK, int 900 assigned to long
int i6 = 900L;        // Error: Cannot assign long to int
\end{lstlisting}
\end{frame}

\begin{frame}[fragile]
\frametitle{Floating-Point Types}
\begin{itemize}
    \item Numbers with a decimal point (or scientific notation).
    \item `float` requires an `f` suffix.
\end{itemize}
\begin{lstlisting}[language=Java]
float f1 = 3.14f;
float f2 = 3.14e38f; // 3.14 x 10^38
float f3 = 1.0;      // Error: 1.0 is a double

double d = 1.79e308;
double d2 = -1.79e308;
double d3 = 4.9e-324; // 4.9 x 10^-324
\end{lstlisting}
\end{frame}

\begin{frame}[fragile]
\frametitle{Boolean Type}
\begin{itemize}
    \item Only two values: `true` and `false`.
    \item Often the result of relational operations.
    \item Stored as 4-byte integers in the JVM.
\end{itemize}
\begin{lstlisting}[language=Java]
boolean b1 = true;
boolean b2 = false;
boolean isGreater = 5 > 3; // true
int age = 12;
boolean isAdult = age >= 18; // false
\end{lstlisting}
\end{frame}

\begin{frame}[fragile]
\frametitle{Character Type}
\begin{itemize}
    \item Represents a single character.
    \item Uses single quotes `'`.
    \item Can represent Unicode characters.
\end{itemize}
\begin{lstlisting}[language=Java]
public class Main {
    public static void main(String[] args) {
        char a = 'A';
        char zh = '\u4E2D';
        System.out.println(a);   // A
        System.out.println(zh);  // \u4E2D
    }
}
\end{lstlisting}
\end{frame}

\begin{frame}[fragile]
\frametitle{Reference Types}
\begin{itemize}
    \item All non-primitive types are reference types.
    \item Store an address pointing to an object in memory.
    \item `String` is a common reference type.
\end{itemize}
\begin{lstlisting}[language=Java]
String s = "hello"; // s is a reference type variable
\end{lstlisting}
\end{frame}

\begin{frame}[fragile]
\frametitle{Constants}
\begin{itemize}
    \item Defined with the `final` modifier.
    \item Cannot be reassigned after initialization.
    \item Use uppercase names by convention.
    \item Avoid "magic numbers" in code.
\end{itemize}
\begin{lstlisting}[language=Java]
final double PI = 3.14; // Constant
double r = 5.0;
double area = PI * r * r;
PI = 300; // Compile error!
\end{lstlisting}
\end{frame}

\begin{frame}[fragile]
\frametitle{The `var` Keyword}
\begin{itemize}
    \item Lets the compiler infer the variable type.
    \item Makes code less verbose.
\end{itemize}
\begin{lstlisting}[language=Java]
// Without var
StringBuilder sb = new StringBuilder();

// With var
var sb = new StringBuilder(); // Compiler infers StringBuilder
\end{lstlisting}
\end{frame}

\begin{frame}[fragile]
\frametitle{Variable Scope}
\begin{itemize}
    \item Scope is defined by `{ ... }` blocks.
    \item Variable is accessible from its point of definition to the end of its block.
    \item Minimize variable scope.
    \item Avoid reusing variable names in nested scopes.
\end{itemize}
\begin{lstlisting}[language=Java]
{
    int i = 0;
    {
        int x = 1;
        {
            String s = "hello";
        } // s scope ends
        String s = "hi"; // OK, different variable
    } // x and s scope end
} // i scope ends
\end{lstlisting}
\end{frame}

\begin{frame}[fragile]
\frametitle{Summary}
\begin{itemize}
    \item Two variable types: \textbf{primitive} and \textbf{reference}.
    \item Primitive types: integers, floats, booleans, characters.
    \item Variables can be \textbf{reassigned}. `=` is assignment, not equality.
    \item \textbf{Constants} (`final`) cannot be reassigned.
    \item Use \textbf{`var`} for cleaner code when the type is obvious.
    \item Define variables in the \textbf{smallest possible scope}.
\end{itemize}
\end{frame}

\section{Integer Operations}
\begin{frame}[fragile]
\frametitle{Integer Operations}
\begin{itemize}
    \item Follow arithmetic rules with nested parentheses
    \item Integer operations are always exact
\end{itemize}
\begin{lstlisting}[language=Java]
public class Main {
    public static void main(String[] args) {
        int i = (100 + 200) * (99 - 88); // 3300
        int n = 7 * (5 + (i - 9)); // 23072
        System.out.println(i);
        System.out.println(n);
    }
}
\end{lstlisting}
\end{frame}

\begin{frame}[fragile]
\frametitle{Division and Remainder}
\begin{itemize}
    \item Integer division yields integer part only
    \item Remainder operation uses `\%`
\end{itemize}
\begin{lstlisting}[language=Java]
int x = 12345 / 67; // 184
int y = 12345 \% 67; // 17 (remainder)

// Division by zero causes runtime error
int z = 100 / 0; // ArithmeticException
\end{lstlisting}
\end{frame}

\begin{frame}[fragile]
\frametitle{Overflow}
\begin{itemize}
    \item Occurs when result exceeds integer range
    \item No error thrown, produces unexpected results
\end{itemize}
\begin{lstlisting}[language=Java]
public class Main {
    public static void main(String[] args) {
        int x = 2147483640;
        int y = 15;
        int sum = x + y;
        System.out.println(sum); // -2147483641
    }
}
\end{lstlisting}
Use \texttt{long} for larger range:
\begin{lstlisting}[language=Java]
long x = 2147483640;
long y = 15;
long sum = x + y; // 2147483655
\end{lstlisting}
\end{frame}

\begin{frame}[fragile]
\frametitle{Compound Assignment Operators}
\begin{itemize}
    \item Shorthand operators for common operations
\end{itemize}
\begin{lstlisting}[language=Java]
int n = 3300;
n += 100; // n = n + 100; -> 3400
n -= 100; // n = n - 100; -> 3300
n *= 2;   // n = n * 2;   -> 6600
n /= 3;   // n = n / 3;   -> 2200
n %= 100; // n = n % 100; -> 0
\end{lstlisting}
\end{frame}

\begin{frame}[fragile]
\frametitle{Increment/Decrement Operators}
\begin{itemize}
    \item \texttt{++} increments, \texttt{--} decrements
    \item Position matters: prefix vs postfix
\end{itemize}
\begin{lstlisting}[language=Java]
int n = 3300;
n++; // 3301 (postfix: use then increment)
++n; // 3302 (prefix: increment then use)

int a = n++; // a = 3302, n = 3303
int b = ++n; // n = 3304, b = 3304

// Avoid complex expressions with ++/--
\end{lstlisting}
\end{frame}

\begin{frame}[fragile]
\frametitle{Bitwise Shift Operations}
\begin{itemize}
    \item Left shift: \texttt{\textless{}\textless{}}, Right shift: \texttt{\textgreater{}\textgreater{}}
    \item Unsigned right shift: \texttt{\textgreater{}\textgreater{}\textgreater{}}
\end{itemize}
\begin{lstlisting}[language=Java]
int n = 7;       // 00000111 = 7
int a = n << 1;  // 00001110 = 14
int b = n << 2;  // 00011100 = 28
int c = n >> 1;  // 00000011 = 3
int d = n >> 2;  // 00000001 = 1

int neg = -536870912;
int e = neg >> 1;  // preserves sign bit
int f = neg >>> 1; // fills with 0
\end{lstlisting}
\end{frame}

\begin{frame}[fragile]
\frametitle{Bitwise Operations}
\begin{itemize}
    \item AND: \texttt{\&}, OR: \texttt{|}, NOT: \texttt{\~}, XOR: \texttt{\^}
\end{itemize}
\begin{lstlisting}[language=Java]
int i = 167776589; // 00001010 00000000 00010001 01001101
int n = 167776512; // 00001010 00000000 00010001 00000000

int and = i & n;   // 167776512
int or = i | n;    // 167776589  
int xor = i ^ n;   // 77
int not = ~i;      // -167776590
\end{lstlisting}
\end{frame}

\begin{frame}[fragile]
\frametitle{Operator Precedence}
\begin{itemize}
    \item Highest: \texttt{()}
    \item \texttt{!}, \texttt{\~}, \texttt{++}, \texttt{--}
    \item \texttt{*}, \texttt{/}, \texttt{\%}
    \item \texttt{+}, \texttt{-}
    \item \texttt{<<}, \texttt{>>}, \texttt{>>>}
    \item \texttt{\&}
    \item \texttt{|}
    \item Lowest: \texttt{+=}, \texttt{-=}, \texttt{*=}, \texttt{/=}
\end{itemize}
\begin{lstlisting}[language=Java]
// Use parentheses for clarity
int result = (a + b) * (c - d) / e;
\end{lstlisting}
\end{frame}

\begin{frame}[fragile]
\frametitle{Type Promotion and Casting}
\begin{itemize}
    \item Operations promote to larger type
    \item Explicit casting may cause data loss
\end{itemize}
\begin{lstlisting}[language=Java]
short s = 1234;
int i = 123456;
int x = s + i; // promoted to int
// short y = s + i; // compilation error

short y = (short)(s + i); // explicit cast
// Risk of incorrect results if value too large
\end{lstlisting}
\end{frame}

\begin{frame}[fragile]
\frametitle{Casting Risks}
\begin{itemize}
    \item Casting large values produces unexpected results
\end{itemize}
\begin{lstlisting}[language=Java]
public class Main {
    public static void main(String[] args) {
        int i1 = 1234567;
        short s1 = (short) i1; // -10617
        System.out.println(s1);
        
        int i2 = 12345678;
        short s2 = (short) i2; // 24910
        System.out.println(s2);
    }
}
\end{lstlisting}
\end{frame}

\begin{frame}[fragile]
\frametitle{Exercise: Sum of First N Numbers}
Calculate sum using formula: $\frac{(1+N) \times N}{2}$
\begin{lstlisting}[language=Java]
public class Main {
    public static void main(String[] args) {
        int n = /*TODO*/;
        int sum = /*TODO*/;
        System.out.println(sum); // 5050
        System.out.println(sum == 5050 ? 
            "Test passed" : "Test failed");
    }
}
\end{lstlisting}
\end{frame}

\begin{frame}[fragile]
\frametitle{Exercise: Sum of First N Numbers (Cont.)}
Calculate sum using formula: $\frac{(1+N) \times N}{2}$
\begin{lstlisting}[language=Java]
public class Main {
    public static void main(String[] args) {
        int n = 100;
        int sum = (1 + n) * n / 2;
        System.out.println(sum); // 5050
        System.out.println(sum == 5050 ? 
            "Test passed" : "Test failed");
    }
}
\end{lstlisting}
\end{frame}

\begin{frame}[fragile]
\frametitle{Summary}
\begin{itemize}
    \item Integer operations are precise but watch for overflow
    \item Use appropriate types (\texttt{int} vs \texttt{long})
    \item Compound assignment operators simplify code
    \item Be careful with casting and type promotion
    \item Use parentheses for complex expressions
    \item Avoid \texttt{byte}/\texttt{short} for arithmetic unless necessary
\end{itemize}
\end{frame}

\section{Float Operations}
\begin{frame}[fragile]
\frametitle{Floating Point Characteristics}
\begin{itemize}
    \item Limited operations: +, -, *, /
    \item Cannot perform bitwise or shift operations
    \item Often cannot be represented precisely
\end{itemize}
\begin{lstlisting}[language=Java]
// Precision issues
double x = 1.0 / 10;    // 0.1 (approximately)
double y = 1 - 9.0 / 10; // 0.1 (approximately)
System.out.println(x);   // 0.1
System.out.println(y);   // 0.09999999999999998
\end{lstlisting}
\end{frame}

\begin{frame}[fragile]
\frametitle{Comparing Floating Point Numbers}
\begin{itemize}
    \item Never use == for floating point comparison
    \item Compare absolute difference against threshold
\end{itemize}
\begin{lstlisting}[language=Java]
double x = 1.0 / 10;
double y = 1 - 9.0 / 10;

double diff = Math.abs(x - y);
if (diff < 0.00001) {
    System.out.println("Equal");
} else {
    System.out.println("Not equal");
}
\end{lstlisting}
\end{frame}

\begin{frame}[fragile]
\frametitle{Type Promotion}
\begin{itemize}
    \item Integers automatically promoted to floating point
    \item Watch for integer division in mixed expressions
\end{itemize}
\begin{lstlisting}[language=Java]
int n = 5;
double d1 = 1.2 + 24.0 / n; // 6.0
double d2 = 1.2 + 24 / n;   // 5.2

System.out.println(d1);
System.out.println(d2);
\end{lstlisting}
\end{frame}

\begin{frame}[fragile]
\frametitle{Special Values}
\begin{itemize}
    \item Division by zero returns special values
    \item No exception thrown
\end{itemize}
\begin{lstlisting}[language=Java]
double d1 = 0.0 / 0;   // NaN
double d2 = 1.0 / 0;   // Infinity
double d3 = -1.0 / 0;  // -Infinity

System.out.println(d1); // NaN
System.out.println(d2); // Infinity
System.out.println(d3); // -Infinity
\end{lstlisting}
\end{frame}

\begin{frame}[fragile]
\frametitle{Explicit Casting}
\begin{itemize}
    \item Fractional part discarded during casting
    \item Values beyond range return max integer
    \item Add 0.5 for rounding
\end{itemize}
\begin{lstlisting}[language=Java]
int n1 = (int) 12.3;   // 12
int n2 = (int) 12.7;   // 12
int n3 = (int) -12.7;  // -12
int n4 = (int) 1.2e20; // 2147483647

// Rounding
double d = 2.6;
int rounded = (int) (d + 0.5); // 3
\end{lstlisting}
\end{frame}

\begin{frame}[fragile]
\frametitle{Exercise: Quadratic Equation}
Solve $ax^2 + bx + c = 0$ using quadratic formula:
\begin{lstlisting}[language=Java]
public class Main {
    public static void main(String[] args) {
        double a = 1.0, b = 3.0, c = -4.0;
        // TODO: Calculate roots
        double r1 = 0;
        double r2 = 0;
        
        System.out.println(r1); // 1.0
        System.out.println(r2); // -4.0
    }
}
\end{lstlisting}
\end{frame}

\begin{frame}[fragile]
\frametitle{Exercise: Quadratic Equation (Cont.)}
Solve $ax^2 + bx + c = 0$ using quadratic formula:
\begin{lstlisting}[language=Java]
public class Main {
    public static void main(String[] args) {
        double a = 1.0, b = 3.0, c = -4.0;
        double discriminant = b * b - 4 * a * c;
        double r1 = (-b + Math.sqrt(discriminant)) / (2 * a);
        double r2 = (-b - Math.sqrt(discriminant)) / (2 * a);
        
        System.out.println(r1); // 1.0
        System.out.println(r2); // -4.0
    }
}
\end{lstlisting}
\end{frame}

\begin{frame}[fragile]
\frametitle{Summary}
\begin{itemize}
    \item Floating point numbers often cannot be represented precisely, and their arithmetic results may have errors.
    \item To compare two floating point numbers, compare the absolute value of their difference against a specific threshold.
    \item When integers and floating point numbers are involved in arithmetic, integers are automatically promoted to floating point numbers.
    \item Floating point numbers can be explicitly cast to integers, but values exceeding the range will always return the maximum integer value.
\end{itemize}
\end{frame}

\section{Boolean Operations}
\begin{frame}[fragile]
\frametitle{Boolean Operators}
\begin{itemize}
    \item Comparison: $>$, $>=$, $<$, $<=$, $==$, $!=$
    \item Logical: $\&\&$ (AND), $||$ (OR), $!$ (NOT)
\end{itemize}
\begin{lstlisting}[language=Java]
int age = 12;
boolean isGreater = 5 > 3;        // true
boolean isZero = age == 0;        // false
boolean isNonZero = !isZero;      // true
boolean isAdult = age >= 18;      // false
boolean isTeen = age > 6 && age < 18; // true
\end{lstlisting}
\end{frame}

\begin{frame}[fragile]
\frametitle{Short-Circuit Evaluation}
\begin{itemize}
    \item Stops evaluating when result is determined
    \item Prevents unnecessary computations
\end{itemize}
\begin{lstlisting}[language=Java]
boolean b = 5 < 3; // false
// Division by zero avoided due to short-circuit
boolean result = b && (5 / 0 > 0);
System.out.println(result); // false

// This would throw ArithmeticException
// boolean result2 = true && (5 / 0 > 0);
\end{lstlisting}
\end{frame}

\begin{frame}[fragile]
\frametitle{Ternary Operator}
\begin{itemize}
    \item Syntax: condition ? expr1 : expr2
    \item Returns expr1 if true, expr2 if false
\end{itemize}
\begin{lstlisting}[language=Java]
int n = -100;
int x = n >= 0 ? n : -n; // Absolute value
System.out.println(x); // 100

// Types must match
String result = n > 0 ? "positive" : "negative";
\end{lstlisting}
\end{frame}

\begin{frame}[fragile]
\frametitle{Exercise: Primary Student Check}
Check if age is 6-12 years:
\begin{lstlisting}[language=Java]
public class Main {
    public static void main(String[] args) {
        int age = 7;
        boolean isPrimaryStudent = ???;
        System.out.println(isPrimaryStudent ? "Yes" : "No");
    }
}
\end{lstlisting}
\end{frame}
\begin{frame}[fragile]
\frametitle{Exercise: Primary Student Check (Cont.)}
Check if age is 6-12 years:
\begin{lstlisting}[language=Java]
public class Main {
    public static void main(String[] args) {
        int age = 7;
        boolean isPrimaryStudent = age >= 6 && age <= 12;
        System.out.println(isPrimaryStudent ? "Yes" : "No");
    }
}
\end{lstlisting}
\end{frame}

\begin{frame}[fragile]
\frametitle{Summary}
\begin{itemize}
    \item AND and OR operations are short-circuit operations.
    \item The ternary operation \texttt{b ? x : y} requires the types of \texttt{x} and \texttt{y} to be the same.
    \item The ternary operation is also a "short-circuit operation," evaluating only \texttt{x} or \texttt{y}.
\end{itemize}
\end{frame}

\section{Characters and Strings}
\begin{frame}[fragile]
\frametitle{Character Type (char)}
\begin{itemize}
    \item Primitive type, holds single Unicode character
    \item 2 bytes per character
\end{itemize}
\begin{lstlisting}[language=Java]
char c1 = 'A';
char c2 = '\u4e2d'; // Chinese character

int n1 = 'A';  // 65
int n2 = '\u4e2d'; // 20013

System.out.println(c1); // A
System.out.println(n2); // 20013
\end{lstlisting}
\end{frame}

\begin{frame}[fragile]
\frametitle{String Type (String)}
\begin{itemize}
    \item Reference type, uses double quotes
    \item Escape sequences for special characters
\end{itemize}
\begin{lstlisting}[language=Java]
String s1 = ""; // Empty string
String s2 = "A"; // 1 character
String s3 = "ABC"; // 3 characters

// Escape sequences
String s4 = "abc\"xyz"; // Contains: a, b, c, ", x, y, z
String s5 = "abc\\xyz"; // Contains: a, b, c, \, x, y, z
String s6 = "Line1\nLine2"; // Newline
\end{lstlisting}
\end{frame}

\begin{frame}[fragile]
\frametitle{String Concatenation}
\begin{itemize}
    \item + operator automatically converts types
    \item Other types converted to strings
\end{itemize}
\begin{lstlisting}[language=Java]
String s1 = "Hello";
String s2 = "world";
String s = s1 + " " + s2 + "!"; // Hello world!

int age = 25;
String info = "Age: " + age; // Age: 25

double pi = 3.14159;
String msg = "PI = " + pi; // PI = 3.14159
\end{lstlisting}
\end{frame}

\begin{frame}[fragile]
\frametitle{Multi-line Strings (Text Blocks)}
\begin{itemize}
    \item Java 13+ feature using """..."""
    \item Common leading spaces removed
\end{itemize}
\begin{lstlisting}[language=Java]
String query = """
               SELECT * FROM users
               WHERE age > 18
               ORDER BY name
               """;

// Equivalent to:
String oldStyle = "SELECT * FROM users\n" +
                  "WHERE age > 18\n" +
                  "ORDER BY name";
\end{lstlisting}
\end{frame}

\begin{frame}[fragile]
\frametitle{String Immutability}
\begin{itemize}
    \item String content cannot be changed
    \item Variables can reference different strings
\end{itemize}
\begin{lstlisting}[language=Java]
String s = "hello";
String t = s;       // t references "hello"
s = "world";        // s now references "world"

System.out.println(t); // hello (unchanged)
System.out.println(s); // world
\end{lstlisting}
\end{frame}

\begin{frame}[fragile]
\frametitle{Null vs Empty String}
\begin{itemize}
    \item null: no object reference
    \item "": empty string object
\end{itemize}
\begin{lstlisting}[language=Java]
String s1 = null;    // No string object
String s2 = "";      // Empty string object
String s3 = s1;      // Also null

System.out.println(s1 == null); // true
System.out.println(s2.isEmpty()); // true
// System.out.println(s1.isEmpty()); // NullPointerException
\end{lstlisting}
\end{frame}

\begin{frame}[fragile]
\frametitle{Exercise: Unicode to String}
Convert Unicode codes to string:
\begin{lstlisting}[language=Java]
public class Main {
    public static void main(String[] args) {
        int a = 72;    // 'H'
        int b = 105;   // 'i'
        int c = 65281; // '!'
        // FIXME:
        String s = a + b + c;
        System.out.println(s); // Hi!
    }
}
\end{lstlisting}
\end{frame}

\begin{frame}[fragile]
\frametitle{Exercise: Unicode to String (Cont.)}
Convert Unicode codes to string:
\begin{lstlisting}[language=Java]
public class Main {
    public static void main(String[] args) {
        int a = 72;    // 'H'
        int b = 105;   // 'i'
        int c = 65281; // '!'
        
        String s = "" + (char)a + (char)b + (char)c;
        System.out.println(s); // Hi!
    }
}
\end{lstlisting}
\end{frame}

\begin{frame}[fragile]
\frametitle{Summary}
\begin{itemize}
    \item Java's character type `char` is a primitive type, while the string type `String` is a reference type.
    \item Primitive type variables "hold" a value, while reference type variables "point to" an object.
    \item Reference type variables can be null.
    \item Distinguish between the null value `null` and the empty string `""`.
\end{itemize}
\end{frame}

\end{document}